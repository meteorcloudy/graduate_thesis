% !Mode:: "TeX:UTF-8"

\begin{Cabstract}{进化树}{rSPR距离}{最大一致森林}{固定参数算法}{基因水平转移}
因为生物进化过程中存在的杂交事件,当研究一组给定物种的不用基因时,可能会得到针对同一组物种的两棵不同但联系紧密的进化树。这种情况下,生物学家需要比较两棵进化树的结构差异。子树剪切再接距离(rSPR距离)正是衡量这种差异的标准之一,许多学者和研究人员已经开发出很多计算两棵进化树rSPR距离的算法和软件工具。计算rSPR距离已经被证明是NP难的。在此之前,计算rSPR距离最快的固定参数算法的时间复杂度为$O(2.35^kn)$,其中$k$是两棵进化树$T_1$和$T_2$之间的rSPR距离,$n$是$T_1$(或$T_2$)的叶节点数。本文提出一种相对更简单但速度更快的固定参数算法,其时间复杂度为$O(2.27^kn)$。
\end{Cabstract}
