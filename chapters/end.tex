% !Mode:: "TeX:UTF-8"
% !Mode:: "TeX:UTF-8"

\chapter{总结及展望}
本文通过对进化树结构对比问题的深入分析,发现了该问题的一些新的性质。基于这些性质并结合前人的研究成果,开辟了研究该问题的新思路,并设计出更加高效的固定参数算法。同时,给出了该算法的关键步骤的证明以及具体的实现方法。最后,使用随机生成的数据和真是的生物数据验证了算法的改进效果,得到了预期的结果。

作者也通过本课题的研究、学习、设计和程序实现,了解并掌握了算法研究的基本方法,锻炼了自身的学术研究能力。在此过程中深深体会到算法研究所需要的艰辛、细致和严谨。

在未来,还可以通过各个方面改进此问题的解决方法。一方面,可以通过改进数据结构设计和算法的实现过程实现使算法时间复杂度的常数进一步降低。另一方面,也可以寻找进化树更多有价值的性质,开辟新的研究思路。同时,也可以结合近似算法和启发式算法进行一些可能的改进。

rSPR距离和最大一致森林(MAF)实际上与杂交数(hybridization number)和最大一直非循环森林(MAFF)等概念有紧密联系。可以借鉴与本文相同的思路对此类相关问题进行研究,这同样也是一个极具价值的研究方向。