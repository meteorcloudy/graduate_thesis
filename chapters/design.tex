% !Mode:: "TeX:UTF-8"
% !Mode:: "TeX:UTF-8"

\chapter{算法思想与设计}
本文的算法与\cite{chen2012faster}中的相似,都是基于Whidden的参数算法\citeup{whidden2010fast}改进而来。因此,在这里有必要先简单介绍Whidden算法的内容,并给出简要的时间复杂度分析。之后,会详细介绍本文的改进思路,并给出具体的方法和步骤。

\section{Whidden的参数算法\citeup{whidden2010fast}}
\subsection{算法概述} \label{whidden_algo}

Whidden的算法所解决的正是本文在第\ref{problem}节中所提出的问题。设$MAF(T_1,F_2,k)$\\为针对此问题的判定函数,$T_1$代表第一棵进化树,$F_2$代表$T_2$删除某个边集后所得的森林,$k$是一个非负整数。若$F_2$能够在删除至多$k$条边后变成$T_1,T_2$的最大一致森林,则$MAF(T_1,F_2,k)$返回$true$,否则返回$false$。初始时,$F_2 = T_2$。求解rSPR距离,只需要从$0$开始枚举$k$的值,直到$MAF(T_1,T_2,k)$返回$true$。因为时间复杂度是关于$k$的指数,所以相对于计算$MAF(T_1,T_2,k)$,计算rSPR距离只会在时间复杂度的常数上有所增加。算法采用递归的思想,$MAF$函数的具体步骤如下:

\begin{enumerate}
	\item 如果$k<0$,返回$false$
	\item \label{step2}如果在$T_1$中存在一对兄弟叶节点$a,b$,并且它们在$F_2$中的对应节点也是兄弟叶节点,那么就合并$a,b$,然后把它们的在$T_1,F_2$中的父节点作为对应的叶节点。重复此步骤,直至没有节点可以合并。
	\item 如果在$F_2$中存在只有一个节点的子树$x$,则将$x$从$T_1,F_2$中移除。重复此步骤,直至没有节点可以移除。此时如果产生新的可以合并的兄弟叶节点,则转至\ref{step2},否则继续。
	\item 如果$F_2$为空,返回$true$
	\item \label{step5}任意选择$T_1$中的一对兄弟叶节点$a,b$(注意到此时$a,b$在$F_2$中一定不是兄弟叶节点),分三种情况讨论(如图\ref{3cases}):
	\begin{enumerate}
		\item Case 1\label{case1}:若$a,b$在$F_2$中属于不同的连通分量。可以证明\citeup{whidden2013fixed},$a$的父边$e_a$和$b$的父边$e_b$两条边中至少有一条需要删除。对于两种情况下修改后的$F_2' = F_2 - \{e_a\}$和$F_2' = F_2 - \{e_b\}$,分别调用两次$MAF(T_1,F_2',k-1)$。只要有任意一次结果为$true$,则返回$true$,否则返回$false$。
		\item Case 2\label{case2}:若$a,b$在$F_2$中连通,并且$a,b$之间有且只有一个悬挂节点$p$。可以证明\citeup{whidden2013fixed},一定存在最优解$E \subset E_{T_2}$,使得$T_2 - E$是$T_1,T_2$的MAF,并且$e_p \in E$。因此,只需要直接修改$F_2$,得到$F_2' = F_2-\{e_p\}$。若$MAF(T_1,F_2',k-1)$结果为$true$,则返回$true$,否则返回$false$。
		\item Case 3\label{case3}:若$a,b$在$F_2$中连通,并且$a,b$之间有至少两个悬挂节点。设$a,b$之间所有悬挂节点的集合为$P=\{p_1,p_2,...,p_m\},m \ge 2$。可以证明\citeup{whidden2013fixed},一定存在最优解$E \subset E_{T_2}$,使得$T_2 - E$是$T_1,T_2$的MAF,并且$e_a \in E$或者$e_b \in E$或者对于所有$1 \le i \le m$,有$e_{p_i} \in E$。因此,分别修改$F_2$得到$F_2' = F_2 - \{e_a\}$、$F_2' = F_2 - \{e_b\}$和$F_2' = F_2 - E_P$($E_P$为P中所有节点对应的父边的集合),然后分别调用两次$MAF(T_1,F_2',k-1)$和一次$MAF(T_1,F_2',k-m)$。只要有任意一次结果为$true$,则返回$true$,否则返回$false$。
	\end{enumerate}  
\end{enumerate}
\pic[htbp]{Whidden算法第\ref{step5}步的三种情况}{width=0.74\textwidth}{3cases}

\subsection{时间复杂度分析}

不难看出,适当地设计数据结构,Whidden算法中前四步可以在$O(n)$的时间内完成。因此,我们关键需要分析搜索树的大小,而搜索树的大小与参数$k$有直接关系。设$C(k)$为参数为$k$时最坏情况下搜索树的大小。根据第\ref{step5}步中的三种情况,我们可以得到下面三个不等式:
\begin{enumerate}
	\item 若$a,b$在$F_2$中属于不同的连通分量。
	\begin{equation*}
		C(k) \le 2C(k-1)
	\end{equation*}
	\item 若$a,b$在$F_2$中连通,并且$a,b$之间有且只有一个悬挂节点。
	\begin{equation*}
		C(k) \le C(k-1)
	\end{equation*}
	\item 若$a,b$在$F_2$中连通,并且$a,b$之间有至少两个悬挂节点。
	\begin{equation*}
		C(k) \le 2C(k-1) + C(k-m),~m \ge 2
	\end{equation*}
\end{enumerate}

可以看出,$C(k)$的最坏情况出现在第3种情况$m = 2$时,此时有
\begin{equation*}
C(k) \le 2C(k-1) + C(k-2)
\end{equation*}

设$C(k)=\alpha ^ k$,带入可得
\begin{equation*}
 1 = 2 \alpha ^ {-1} + \alpha ^ {-2}
\end{equation*}

解得$\alpha = 1 + \sqrt{2} \approx 2.42$,因此Whidden参数算法复杂度为$O(2.42^kn)$。Whidden的算法最大贡献在于发现了Case 2中当$a,b$直接只有$1$个悬挂节点便可以直接删除其父边的规律。而该算法的限制则在于Case 3,本文正是试图寻找方法改进Case 3的最坏情况,具体内容将在下一节讨论。

\section{改进思路}
显然,当叶节点个数$n$小于$3$时,两棵进化树的rSPR距离一定是$0$,因此在后面的讨论中,我们假设$n \ge 3$。为了改进Whidden的算法,我们试图在Whidden所利用的两个兄弟叶节点的基础之上加入更多的节点,以此获取更多的信息用以改进算法。一种自然的想法是在$T_1$中寻找形如图\ref{3and4}中(a)的结构。但(a)所表示的结构并非在所有进化树中都会出现,例如一棵叶节点个数为$4$的满二叉树。然而可以证明,若(a)中的结构不存在,一定存在形如图\ref{3and4}中(b)所表示的结构。
\clearpage
\pic[htbp]{两种结构}{width=0.7\textwidth}{3and4}

\begin{yinli}\label{yinli1}
对于任意一棵叶节点数大于$3$的进化树$T$,存在下面至少一种情况:
\begin{itemize}
	\item 存在$T$的三个叶节点$a,b,c$,满足$a,b$是兄弟节点,并且$a,b$的父节点与$c$是兄弟节点(图\ref{3and4}~(a))
	\item 存在$T$的四个叶节点$a,b,c,d$,满足$a,b$是兄弟节点,$c,d$是兄弟节点,并且$a,b$的父节点与$c,d$的父节点是兄弟节点(图\ref{3and4}~(b))
\end{itemize}
\end{yinli}

\textbf{证明:}不妨设$a$是离$T$的根节点最远的叶节点,根据进化树的定义,一定存在$a$的兄弟节点$b$。(图\ref{3and4_pro}~(a))设$p$为$a,b$的父节点的兄弟节点,因为$a$离根节点的距离最远,那么以$p$为根节点的子树$T_p$只有两种可能。
\begin{itemize}
	\item $p$为叶节点,如图\ref{3and4_pro}~(b),此时出现引理\ref{yinli1}中第一种情况
	\item $p$不为叶节点,则$p$一定有且只有两个叶节点,如图\ref{3and4_pro}~(c),此时出现引理\ref{yinli1}中第二种情况
\end{itemize}

\pic[htbp]{引理\ref{yinli1}证明}{width=0.9\textwidth}{3and4_pro}

于是,基于引理\ref{yinli1}我们可以改进Whidden算法的第\ref{step5}步。

\section{改进算法}
本文改进算法的前四步与第\ref{whidden_algo}节中叙述的完全一致,主要针对第\ref{step5}步进行优化。为了方便叙述,我们作如下两个定义:

\begin{dingyi}
\textbf{设$Q$为节点$a$到节点$b$的路径上所有节点的集合,但不包括$a,b$。有节点$x$,记其父节点为$f_x$,若满足$x \notin Q$且$x \notin \{a,b\}$且$f_x \in Q$,则称$x$为$a,b$之间的悬挂节点。}
\end{dingyi}

\begin{dingyi}
\textbf{记$P(a,b)$为节点$a$到节点$b$的路径上所有悬挂节点的集合,$E_P$表示集合$P$中所有节点父边的集合。}
\end{dingyi}


在详细叙述本文的改进方法之前,我们再明确一下当$T_1,F_2$经过前四步的处理后所具有的两点性质:

\begin{enumerate}
	\item $T_1$中的任意一对兄弟叶节点$a,b$在$F_2$中一定不是兄弟节点。换句话说,$a,b$在$F_2$中要么不连通,要么连通但$|P(a,b)| \ge 1$。
	\item \label{2point} $T_1,F_2$中的任意一个叶节点$x$都不可能是某棵树的根节点。这意味着任意叶节点$x$的父边$e_x$一定存在。
\end{enumerate}

改进算法的第五步首先是在$T_1$中寻找图\ref{3and4_pro}~(a)所表示的结构,这可以通过遍历$T_1$在最多$O(n)$的时间内完成。接下来,我们根据子树$T_p$的两种情况分类讨论。

\subsection{当$p$为叶节点}
\textit{Case 1: }当$p$为叶节点时,我们得到$T_1$中的$3$个叶节点$a,b,c$,使其结构如图\ref{3node}。此时,根据$a,b,c$在$F_2$中的不同形态,我们采取不同的策略。
\pic[htbp]{\textit{Case 1 }$T_1$中的$a,b,c$}{width=0.25\textwidth}{3node}

\textit{Case 1.1: }若$a,b$在$F_2$中属于不同的连通分量,此时出现Whidden算法第\ref{step5}步中的情况\ref{case1}(简称\textbf{情况\ref{case1}}),用相同方法处理。

\textit{Case 1.2: }若$a,b$在$F_2$中属于相同的连通分量,但$|P(a,b)|=1$时,此时出现Whidden算法第\ref{step5}步中的情况\ref{case2}(简称\textbf{情况\ref{case2}}),也用相同方法处理。

\textit{Case 1.3: }若$a,b$在$F_2$中属于相同的连通分量,且满足$|P(a,b)| \ge 2$,但$a,b$与$c$不属于同一连通分量(如图\ref{case_1_3})。
\pic[htbp]{\textit{Case 1.3 }$F_2$中的$a,b,c$}{width=0.45\textwidth}{case_1_3}

设$x$为节点$a$在最终的MAF中的兄弟节点,根据节点$x$可能的情况,我们进行如下分类讨论:

\begin{enumerate}
	\item 如果$x \in \emptyset$,即$a$最终为孤立节点,那么必须删除$e_a$。删除后在$T_1$中$b,c$为兄弟,而在$F_2$中$b,c$不连通,此时出现\textbf{情况\ref{case1}},可以有两种选择,删除$e_b$或者$e_c$。因此分别调用两次递归函数,$MAF(T_1,F_2-\{e_a,e_b\},k-2)$和$MAF(T_1,F_2-\{e_a,e_c\},k-2)$。
	\item 如果$x = b$,则需删除$F_2$中$a$到$b$之间所有的悬挂节点的父边$E_{P(a,b)}$。因此只需调用一次递归函数,$MAF(T_1,F_2-E_{P(a,b)},k-|P(a,b)|)$。
	\item 如果$x \notin \emptyset$且$x \neq b$,由$T_1$的结构可知必须删除$b$的父边$e_b$。删除后在$T_1$中$a,c$为兄弟,而在$F_2$中$a,c$不连通,此时出现\textbf{情况\ref{case1}},可以有两种选择,删除$e_a$或者$e_c$。因此分别调用两次递归函数,$MAF(T_1,F_2-\{e_a,e_b\},k-2)$和$MAF(T_1,F_2-\{e_b,e_c\},k-2)$。$MAF(T_1,F_2-\{e_a,e_b\},k-2)$与1中重复,可省略。\\
\end{enumerate}
综上,对于\textit{Case 1.3},我们需要调用四次递归函数:
\begin{center}
$MAF(T_1,F_2,k) = \left\{
\begin{array}{lr}
         MAF(T_1,F_2-\{e_a,e_b\},k-2) & \mbox{or}\\ 
         MAF(T_1,F_2-\{e_a,e_c\},k-2) & \mbox{or}\\
         MAF(T_1,F_2-\{e_b,e_c\},k-2) & \mbox{or}\\
         MAF(T_1,F_2-E_{P(a,b)},k-|P(a,b)|) & 
\end{array}
\right.$
\end{center}

\textit{Case 1.4: }若$a,b$在$F_2$中属于相同的连通分量,满足$|P(a,b)| \ge 2$,且$a,b$与$c$属于同一连通分量。则需要进行更细节的讨论。
% \clearpage
$ $\\

\textit{Case 1.4.1: }在\textit{Case 1.4}的前提下,若在$F_2$中存在$|P(a,c)|=0$或者$|P(b,c)|=0$,即$a,c$或者$b,c$是兄弟节点。(如图\ref{case_1_4_1})
\pic[htbp]{\textit{Case 1.4.1 }$F_2$中的$a,b,c$}{width=0.5\textwidth}{case_1_4_1}

因为在$T_1$中有$P(a,c)=\{b\},P(b,c)=\{a\}$,此时出现\textbf{情况\ref{case2}}。对应的,应该删除$e_b$或者$e_a$。
因此,对于\textit{Case 1.4.1}有:
\begin{center}
$MAF(T_1,F_2,k) = \left\{
\begin{array}{lcl}
         MAF(T_1,F_2-\{e_b\},k-1) & \mbox{for} & |P(a,c)|=0\\ 
         MAF(T_1,F_2-\{e_a\},k-1) & \mbox{for} & |P(b,c)|=0\\
\end{array}
\right.$
\end{center}
$  $\\

\textit{Case 1.4.2: }在\textit{Case 1.4}的前提下,若$F_2$中满足$|P(a,c)|=1$且$|P(b,c)|=1$,此时$F_2$中的$a,b,c$只可能是图\ref{case_1_4_2}中的两种情况。
\pic[htbp]{\textit{Case 1.4.2 }$F_2$中的$a,b,c$}{width=0.7\textwidth}{case_1_4_2}

同样,设$x$为节点$a$在最终的MAF中的兄弟节点,根据节点$x$可能的情况,我们首先针对图\ref{case_1_4_2}~(a)的情形进行如下分类讨论:
\begin{enumerate}
	\item 如果$x \in \emptyset$,即$a$最终为孤立节点,那么必须删除$e_a$。删除后在$T_1$中$b,c$为兄弟,而在$F_2$中$P(b,c)=\{p\}$,此时出现\textbf{情况\ref{case2}},因此可以直接删除$b,c$之间的悬挂节点$p$的父边。需调用一次递归函数,$MAF(T_1,F_2-\{e_a,e_p\},k-2)$。
	\item 如果$x = b$,则需删除$F_2$中$a$到$b$之间所有的悬挂节点的父边$E_{P(a,b)}$。注意到此时$P(a,b)=\{c,p\}$,需调用一次递归函数,$MAF(T_1,F_2-\{e_c,e_p\},k-2)$。
	\item 如果$x = c$,则需删除$e_b$和$F_2$中$a$到$c$之间所有的悬挂节点的父边$E_{P(a,c)}$。注意到此时$P(a,c)=\{p\}$,需调用一次递归函数,$MAF(T_1,F_2-\{e_b,e_p\},k-2)$。
	\item 如果$x \notin \emptyset$且$x \notin \{b,c\}$,则需要删除$e_b,e_c$。调用一次递归函数,$MAF(T_1,F_2-\{e_b,e_c\},k-2)$。
\end{enumerate}

然而,通过观察,实际上我们可以发现$x=b$和$x=c$两种情况处理后的$T_1,F_2$实际上是等价的。对于如图\ref{case_1_4_2_eq}的两片森林,只需要简单地交换节点$b,c$的标识即可让它们等价,而这并不会影响最终MAF的大小。
\pic[htbp]{$x=b$和$x=c$两种情况处理后的$F_2$}{width=0.4\textwidth}{case_1_4_2_eq}

因此,我们可以在$x=b$和$x=c$两种选择中省略一种,不妨省略$x=c$。而对于图\ref{case_1_4_2}~(b)的情况,我们交换$a,b$所代表的节点后可以用完全相同的方法处理。

综上,对于\textit{Case 1.4.2},我们需要调用三次递归函数:
\begin{center}
$MAF(T_1,F_2,k) = \left\{
\begin{array}{lr}
         MAF(T_1,F_2-\{e_a,e_p\},k-2) & \mbox{or}\\ 
         MAF(T_1,F_2-\{e_c,e_p\},k-2) & \mbox{or}\\
         MAF(T_1,F_2-\{e_b,e_c\},k-2) & 
\end{array}
\right.$
\end{center}
$ $\\

\textit{Case 1.4.3: }在\textit{Case 1.4}的前提下,排除\textit{Case 1.4.1}和\textit{Case 1.4.2},$F_2$中一定有$|P(a,c)| \ge 1$且$|P(b,c)| \ge 1$且存在$x \in \{a,b\}$使$|P(x,c)| \ge 2$。不妨设$x=a$,综合之前的条件,有
\begin{center}
$\left\{
\begin{array}{l}
	|P(a,b)| \ge 2\\
	|P(a,c)| \ge 2\\
	|P(b,c)| \ge 1
\end{array}
\right.$
\end{center}

在满足上述条件的情况下,$F_2$中$a,b,c$有可能为如图\ref{case_1_4_3}三种形态:
\pic[htbp]{\textit{Case 1.4.3 }$F_2$中的$a,b,c$}{width=0.7\textwidth}{case_1_4_3}

虽然是三种不同形态,实际上我们都可以采取类似\textit{Case 1.4.2}的策略来处理。跟之前一样,设$x$为节点$a$在最终的MAF中的兄弟节点,根据节点$x$可能的情况,进行如下分类:
\begin{enumerate}
	\item 如果$x \in \emptyset$,即$a$最终为孤立节点,那么必须删除$e_a$。需调用一次递归函数,$MAF(T_1,F_2-\{e_a\},k-1)$。
	\item 如果$x = b$,则需删除$F_2$中$a$到$b$之间所有的悬挂节点的父边$E_{P(a,b)}$。需调用一次递归函数,$MAF(T_1,F_2-E_{P(a,b)},k-|P(a,b)|)$。
	\item 如果$x = c$,则需删除$e_b$和$F_2$中$a$到$c$之间所有的悬挂节点的父边$E_{P(a,c)}$,即$E_{P(a,c) \cup \{b\}}$。需调用一次递归函数,$MAF(T_1,F_2-E_{P(a,c) \cup \{b\}},k-|P(a,c) \cup \{b\}|)$。
	\item 如果$x \notin \emptyset$且$x \notin \{b,c\}$,则需要删除$e_b,e_c$。调用一次递归函数,$MAF(T_1,F_2-\{e_b,e_c\},k-2)$。
\end{enumerate}
\clearpage
综上,对于\textit{Case 1.4.3},我们需要调用四次递归函数:
\begin{center}
$MAF(T_1,F_2,k) = \left\{
\begin{array}{lr}
         MAF(T_1,F_2-\{e_a\},k-1) & \mbox{or}\\ 
         MAF(T_1,F_2-E_{P(a,b)},k-|P(a,b)|) & \mbox{or}\\
         MAF(T_1,F_2-E_{P(a,c) \cup \{b\}},k-|P(a,c) \cup \{b\}|) & \mbox{or}\\
         MAF(T_1,F_2-\{e_b,e_c\},k-2) & 
\end{array}
\right.$
\end{center}

为了后文时间复杂度的分析,我们在此先分析一下$|P(a,c) \cup \{b\}|$的大小。

\begin{yinli}\label{yinli2}
\textbf{	
设$a,b,c$为同一进化树上的三个节点,且满足$b \in P(a,c)$。则存在如下关系}
\end{yinli}
\begin{equation*}
\textbf{$|P(a,c)| = |P(a,b)|+|P(b,c)|$}
\end{equation*}

\textbf{证明:}记$R_{xy}$为节点$x,y$的最近公共祖先。如果节点$y$是$x$的祖先,则记为$x \prec y$。根据$R_{ab},R_{ac},R_{bc}$的不同情况可能出现如图\ref{yinli_3_3_1_pro}的两种形态。

首先我们针对$R_{ab} \prec R_{ac}$的情况进行分析。根据该情况下进化树的结构和悬挂节点的定义可得到如下几个等式:

\begin{center}
$\left\{
\begin{array}{lr}
	|P(a,b)| = |P(a,R_{ab})| + |P(R_{ab},b)|\\
	|P(a,c)| = |P(a,R_{ab})| + |P(R_{ab},c)| + 1\\
	|P(b,c)| = |P(b,R_{ab})| + |P(R_{ab},c)| + 1\\
\end{array}
\right.$
\end{center}
又因为节点$b$与$R_{ab}$直接相连,可知$P(b,R_{ab}) =P(R_{ab},b) = 0$。带入之前的三个等式即可得到:
\begin{equation*}
|P(a,c)| = |P(a,b)|+|P(b,c)|
\end{equation*}

同理,对于$R_{bc} \prec R_{ac}$的情况,可以用完全相同的方法证明。
\pic[htbp]{当$b \in P(a,c)$时$a,b,c$的可能形态}{width=0.55\textwidth}{yinli_3_3_1_pro}

\clearpage

对于$|P(a,c) \cup \{b\}|$的大小,我们考虑$b$是否属于$P(a,c)$
\begin{itemize}
	\item 当$b \notin P(a,c)$,则有$|P(a,c) \cup \{b\}| = |P(a,c)|+1$。\\
		而根据\textit{Case 1.4.3}的条件$|P(a,c)| \ge 2$,因此$|P(a,c) \cup \{b\}| \ge 3$。
	\item 当$b \in P(a,c)$,则有$|P(a,c) \cup \{b\}| = |P(a,c)|$。\\
		但由引理\ref{yinli2},可得$|P(a,c)| = |P(a,b)| + |P(b,c)|$。\\
		而根据\textit{Case 1.4.3}的条件$|P(a,b)| \ge 2$且$P(b,c) \ge 1$,因此$|P(a,c) \cup \{b\}| \ge 3$。\\
\end{itemize}
 
综上,在\textit{Case 1.4.3}中,$|P(a,c) \cup \{b\}| \ge 3$。\\


\subsection{当$p$不为叶节点}

\textit{Case 2: }当$p$不为叶节点时,我们得到$T_1$中4个叶节点$a,b,c,d$,使其结构如图\ref{4node}。此时,如果继续完整地讨论$a,b,c,d$在$F_2$中的不同形态将会十分复杂,而本文通过将4点情况转化为3点情况巧妙地避免了很多复杂的讨论。首先我们根据$a,b,c,d$在$F_2$中的连通性进行分类讨论。\\

\pic[htbp]{\textit{Case 2 }$T_1$中的$a,b,c,d$}{width=0.32\textwidth}{4node}

\textit{Case 2.1: }若$a,b$(或$c,d$)在$F_2$中属于不同的连通分量,此时出现Whidden算法第\ref{step5}步中的情况\ref{case1},用相同方法处理。\\

\textit{Case 2.2: }若$a,b$(或$c,d$)在$F_2$中属于相同的连通分量,但$|P(a,b)|=1$(或$|P(c,d)|=1$)时,此时出现Whidden算法第\ref{step5}步中的情况\ref{case2},也用相同方法处理。\\

\textit{Case 2.3: }若$a,b$在$F_2$中连通,$c,d$也在$F_2$中连通,且满足$|P(a,b)| \ge 2,|P(c,d)| \ge 2$,但$a,b$与$c,d$不属于同一连通分量(如图\ref{case_2_3})。
\pic[htbp]{\textit{Case 2.3 }$F_2$中的$a,b,c,d$}{width=0.4\textwidth}{case_2_3}

类似\textit{Case 1}中的分析,设$x$为节点$c$在最终的MAF中的兄弟节点,根据节点$x$可能的情况,进行如下分类讨论:
\begin{enumerate}
	\item 如果$x \neq d$,根据$T_1$中的形态可知$e_c,e_d$必然有至少一个会被删除。若删除$e_d$,那么此时$a,b,c$满足\textit{Case 1.3}的条件。同理,若删除$e_c$,则$a,b,d$也会满足\textit{Case 1.3}的条件。因此,我们只需要调用两次处理\textit{Case 1.3}的函数,$CASE\_1\_3(T_1,F_2-\{e_d\},k-1,a,b,c)$和$CASE\_1\_3(T_1,F_2-\{e_c\},k-1,a,b,d)$。
	\item 如果$x = d$,则需删除$F_2$中$c$到$d$之间所有的悬挂节点的父边$E_{P(c,d)}$。然后只需调用一次递归函数,$MAF(T_1,F_2-E_{P(c,d)},k-2)$。
\end{enumerate}

综上,对于\textit{Case 2.3},我们需要调用3次递归函数:
\begin{center}
$MAF(T_1,F_2,k) = \left\{
\begin{array}{lr}
         CASE\_1\_3(T_1,F_2-\{e_d\},k-1,a,b,c) & \mbox{or}\\ 
         CASE\_1\_3(T_1,F_2-\{e_c\},k-1,a,b,d) & \mbox{or}\\
         MAF(T_1,F_2-E_{P(a,b)},k-2) & \mbox{or}
\end{array}
\right.$
\end{center}
$ $\\

\textit{Case 2.4: }若$a,b,c,d$在$F_2$中都属于同一连通分量,且满足$|P(a,b)| \ge 2,|P(c,d)| \ge 2$。此时,$a,b,c,d$在$F_2$中会有很多种不同形态,但我们可以通过讨论$c,d$在最终MAF的情况,将问题转化为\textit{Case 1.4.3}的情况。

类似\textit{Case 1}中的分析,设$x$为节点$c$在最终的MAF中的兄弟节点,根据节点$x$可能的情况,进行如下分类讨论:
\begin{enumerate}
	\item 如果$x \neq d$,根据$T_1$中的形态可知$e_c,e_d$必然有至少一个会被删除。若删除$e_d$,那么此时$a,b,c$满足\textit{Case 1.4.3}的条件。同理,若删除$e_c$,则$a,b,d$也会满足\textit{Case 1.4.3}的条件。因此,我们只需要调用两次处理\textit{Case 1.4.3}的函数,$CASE\_1\_4\_3(T_1,F_2-\{e_d\},k-1,a,b,c)$和$CASE\_1\_4\_3(T_1,F_2-\{e_c\},k-1,a,b,d)$。
	\item 如果$x = d$,则需删除$F_2$中$a$到$b$之间所有的悬挂节点的父边$E_{P(c,d)}$。然后只需调用一次递归函数,$MAF(T_1,F_2-E_{P(c,d)},k-2)$。
\end{enumerate}

综上,对于\textit{Case 2.4},我们需要调用3次递归函数:
\begin{center}
$MAF(T_1,F_2,k) = \left\{
\begin{array}{lr}
         CASE\_1\_4\_3(T_1,F_2-\{e_d\},k-1,a,b,c) & \mbox{or}\\ 
         CASE\_1\_4\_3(T_1,F_2-\{e_c\},k-1,a,b,d) & \mbox{or}\\
         MAF(T_1,F_2-E_{P(c,d)},k-2) & \mbox{or}
\end{array}
\right.$
\end{center}


\section{改进算法的时间复杂度}

类似Whidden的算法,我们每次递归的函数也可以在$O(n)$的时间内完成。设$C(k)$为参数为k时最坏情况下搜索数的大小。对于\textit{Case 1}中的所有情况以及\textit{Case 2.1, Case 2.2}可以直接列出以下不等式:
\begin{center}
$\left\{
\begin{array}{lr}
	C(k) \le 2C(k-1) & Cases~1.1,~2.1\\
	C(k) \le C(k-1) &  Cases~1.2,~1.4.1,~2.2\\
	C(k) \le 3C(k-2) + C(k-m),~m \ge 2 & Case~1.3 \\
	C(k) \le 3C(k-2) & Case~1.4.2 \\
	C(k) \le C(k-1)+C(k-2) + C(k-m_1) + C(k-m_2),~m_1 \ge 2, m_2 \ge 3 & Case~1.4.3
\end{array}
\right.$
\end{center}

对于$Case 2.3$和$Case 2.4$,由于需要转移到$Case 1.3$和$Case 1.4.3$,我们设$Q_1$为$Case 1.3$所限制的搜索树大小,$Q_2$为$Case 1.4.3$所限制的搜索树大小。根据之前的分析,可以得到不等式:

\begin{center}
$\left\{
\begin{array}{lr}
	C(k) \le 2Q_1(k-1) + C(k-2) & Case~2.3 \\
	C(k) \le 2Q_2(k-1) + C(k-2) & Case~2.4 \\
	Q_1(k) \le 3C(k-2) + C(k-m),~m \ge 2 & Case~1.3 \\
	Q_2(k) \le C(k-1)+C(k-2) + C(k-m_1) + C(k-m_2),~m_1 \ge 2, m_2 \ge 3 & Case~1.4.3
\end{array}
\right.$
\end{center}

我们取\textit{Case 1.3}和\textit{Case 1.4.3}的最坏情况带入,即$m = 2, m_1 = 2, m_2 = 3$时,可以得到
\begin{center}
$\left\{
\begin{array}{lr}
	C(k) \le C(k-2) + 8C(k-3) & Case~2.3 \\
	C(k) \le 3C(k-2) + 4C(k-3) + 2C(k-4)& Case~2.4
\end{array}
\right.$
\end{center}

综合\textit{Case 1}和\textit{Case 2},最差的情况为\textit{Case 2.4}。设$C(k) = \alpha ^ k$,带入可得
\begin{equation*}
 1 = 3\alpha ^ {-2} + 4\alpha ^ {-3} + 2\alpha ^{-4}
\end{equation*}
解得$\alpha \approx 2.27$,因此改进算法的复杂度为$O(2.27^kn)$。






















































