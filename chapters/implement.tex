% !Mode:: "TeX:UTF-8"

\lstset{language=C++,
  showspaces=false,
  showtabs=false,
  % breaklines=true,
  showstringspaces=false,
  breakatwhitespace=true,
  escapeinside={(*@}{@*)},
  commentstyle=\color{greencomments},
  keywordstyle=\color{bluekeywords},
  stringstyle=\color{redstrings},
  basicstyle=\linespread{1} \ttfamily,
  escapechar=`,
  extendedchars=false,
  breaklines=true,
  % frame = trbl,
  numbers=left, 
  % numbersep=2pt,
  numberstyle=\small \color{gray},
  columns=fullflexible
}

\chapter{数据结构设计及算法实现}
为了验证上一章所提出的参数算法的性能,我们需要将算法用一种编程语言实现。考虑到编程语言的复杂性和效率,我们选用C++语言,利用其面向对象的程序设计方法来实现该算法。精心设计的数据结构可以大大降低编程的复杂度并且提高程序的运行效率。因此选取合适的数据结构是本章的首要任务。

\section{数据结构设计}

\subsection{树节点}
在设计树节点的数据结构时,需要考虑到能够方便地访问其父节点,子节点。为了简便与扩展性,将子节点指针用vector存储,实际运行中容器的大小不会超过2。

同时,因为两棵进化树的节点需要进行匹配,如果两棵进化树中存在同构的子树,便将两子树的根节点配对,记录对方的节点id为pairId,若未配对则pairId为-1。

其次,定义此类相关的存取方法和一些辅助方法。\\
\textbf{TreeNode类定义如下}:

\begin{CJK}{UTF8}{gbsn}
\begin{lstlisting}
class TreeNode{
  private:
    TreeNode * parent; // `\color{greencomments}父节点的指针`
    vector<TreeNode *> children; // `\color{greencomments}子节点的指针`
    int id;             // `\color{greencomments}节点id`
    string label;          // `\color{greencomments}叶节点标识,非叶节点则为空`
    int pairId;           // `\color{greencomments}对应节点id`

  public:
    // `\color{greencomments}TreeNode的构造函数与析构函数`
    TreeNode();
    TreeNode(int id);
    TreeNode(int id,int label,int pairId);
    TreeNode(TreeNode *p); // `\color{greencomments}拷贝构造函数`
    ~TreeNode();

    void AddChild(TreeNode *pChild);  // `\color{greencomments}添加子节点`
    void RemoveChild(TreeNode * p);   // `\color{greencomments}删除子节点`
    void SetId(int x);      
    void SetLabel(const int str);
    void SetPairId(const int id);
    int GetId();
    int GetLablel();
    int GetPairId();    
    int GetChildrenSize();
    TreeNode * GetChild(int i);
    TreeNode * GetParent();
    TreeNode * GetRoot();    
    TreeNode * GetSiblingNode();
    bool IsLeaf();  // `\color{greencomments}若是叶节点返回true,否则返回false`
    bool IsRtLeaf();  // `\color{greencomments}若节点已配对则返回true,否则返回false`
    bool IsRoot();  // `\color{greencomments}返回该节点所在子树的根节点`
    bool IsSibling(const TreeNode * p); // `\color{greencomments}判断与另一节点是否是兄弟节点`
    string ToString();  // `\color{greencomments}若是叶节点则返回标识,否则返回空`
};
\end{lstlisting}
\end{CJK}

\subsection{进化树}
考虑到需要同时表示进化树和进化树森林,我们将PhylogenyTree类定义成包含多个根节点的森林,将所有的根节点用vector保存。

算法实现过程中需要通过节点id或label访问节点的操作,用unordered\_map为容器用idMap和labelMap进行映射,unordered\_map内部原理是哈希表,单次映射操作复杂度为$O(1)$。

最后定义各种相关的读写方法和辅助方法,其中比较重要的有Contract、DeleteEdge、DeleteEdges和randomSPR等方法,这些都是进化树的几个基本操作,详见注释。
\\ \textbf{PhylogenyTree类定义如下}:

\begin{CJK}{UTF8}{gbsn}
\begin{lstlisting}
class PhylogenyTree{    
  private:
    vector<TreeNode *> roots;
    unordered_map<int,TreeNode *> idMap;    // `\color{greencomments}节点id映射到节点指针`
    unordered_map<string,TreeNode *> labelMap;  /* `\color{greencomments}节点label映射到节点`
                                                 `\color{greencomments}指针,非叶节点则不存在`*/
    void BuildMaps();   // `\color{greencomments}遍历森林,生成idMap和labelMap`
  public:
    PhylogenyTree();
    PhylogenyTree(PhylogenyTree * tree);   // `\color{greencomments}拷贝构造函数,复制另一棵树`
    ~PhylogenyTree();

    void BuildByNewick(string newickStr);   // `\color{greencomments}根据Newick格式构造进化树`
    TreeNode * GetRootNode(int k);  // `\color{greencomments}获取第k个根节点`
    TreeNode * GetNodeById(int id);
    TreeNode * GetNodeByLabel(int label);
    int GetLabeledNodeNum();    // `\color{greencomments}获取被标记节点数`
    int GetNodeNum();   // `\color{greencomments}获取节点总数`
    int GetRootNum();   // `\color{greencomments}获取根节点数`
    vector<TreeNode *> GetAllNode();    
    vector<TreeNode *> GetAllLabeledNode();
    vector<TreeNode *> GetAllPairedNode();

    void AddRoot(TreeNode * p); // `\color{greencomments}添加根节点,相当于添加一棵子树`
    void RemoveRoot(TreeNode * p);  // `\color{greencomments}删除根节点,相当于删除一棵子树`
    void Contract();    // `\color{greencomments}对所有子树进行收缩操作`
    void DeleteEdge(int nid); // `\color{greencomments}删除某节点的父边`
    void DeleteEdges(vector<int> &nids); // `\color{greencomments}删除一些节点的父边`
    void EraseNode(TreeNode * p);  // `\color{greencomments}将节点从idMap和labelMap中移除`
    
    void randomSPR(int k);  // `\color{greencomments}随机进行k次rSPR操作(生成随机数据用)`
    string ToString();  // `\color{greencomments}将树或森林表示成Newick格式的字符串`
};

\end{lstlisting}
\end{CJK}

\section{算法实现}

从上一章的算法描述部分可以看出,我们的算法实际上是通过限制深度的深度优先搜索算法来寻找解,并且在每次搜索时判断当前状态所属情况,然后进行分类讨论。因此,我们将不同情况分为不同的模块,模块之间可以互相调用,这样即理清了程序的结构也增加了代码重用,使代码结构清晰简洁明了。

同样,为了遵循面相对象的设计方式,我们用FPTsolver类来实现算法的主体。将每种情况用一个方法实现,并定义相关的辅助方法。由于篇幅限制,在这里只给出类FPTsolver的定义和递归主体函数MAF的具体实现方法。MAF方法主要完成了算法的前四步,然后寻找可以处理的一组点进行情况判定,并调用相应的方法进行处理。完整源代码详见\href{https://github.com/meteorcloudy/FPT_faster}{FPT\_faster}。

$ $\\ \textbf{FPTsolver类定义如下}:
\input{code/FPTsolver}
$ $\\\textbf{MAF函数的实现如下}:
\begin{CJK}{UTF8}{gbsn}
\begin{lstlisting}
bool FPTSolver:: MAF(PhylogenyTree * F,PhylogenyTree * T1,
                     PhylogenyTree * F2, int k){
    if (k<0) return false; // `\color{greencomments}第一步`
    while (MergeSiblingNodes(T1, F2) | MoveTree(F, T1, F2)); 
                            // `\color{greencomments}第二、三步`
    if (F2->GetRootNum() == 0) { // `\color{greencomments}第四步`
        ans = new PhylogenyTree(F); return true;
    }
    if (k==0) return false; // `\color{greencomments}此时至少需要删掉一条边,k==0返回false`
    Group grp1 = FindGroup(T1),grp2; //`\color{greencomments}寻找3点组或4点组,由type字段注明`
    grp2.type = grp1.type;
    grp2.a = F2->GetNodeById(grp1.a->GetPairId()); //`\color{greencomments}获取$F_2$中对应节点组`
    grp2.b = F2->GetNodeById(grp1.b->GetPairId());
    grp2.c = F2->GetNodeById(grp1.c->GetPairId());
    if (grp1.type == 4) grp2.d = F2->GetNodeById(grp1.d->GetPairId());
    TreeNode * roota = grp2.a->GetRoot();
    TreeNode * rootb = grp2.b->GetRoot();
    bool res = false;
    if (grp1.type == 3){    // `\color{greencomments}3点情况`
        if (roota != rootb) {
            res = Case_1_1(F,T1,F2,k,grp2);
        } else {
            TreeNode * Rab = F2->GetNodeById(lca.ask(grp2.a->GetId(),
                            grp2.b->GetId())); // `\color{greencomments}获取a,b的最近公共祖先`
            vector<int> pedant_ab;  // `\color{greencomments}获取a,b之间的悬挂节点`
            FindPendantNodes(grp2.a, Rab, pedant_ab);
            FindPendantNodes(grp2.b, Rab, pedant_ab);
            if (pedant_ab.size()==1){
                res = Case_1_2(F, T1, F2, k, pedant_ab[0]);
            } else {
                TreeNode * rootc = grp2.c->GetRoot();
                if (roota != rootc) {
                    res = Case_1_3(F, T1, F2, k, grp2, pedant_ab);
                } else {
                    res = Case_1_4(F, T1, F2, k, grp2, pedant_ab);
                }
            }
        }
    } else {                // `\color{greencomments}4点情况`
        if (roota != rootb) {
            res = Case_2_1(F,T1,F2,k,grp2.a->GetId(),grp2.b->GetId());
            return res;
        }
        TreeNode * rootc = grp2.c->GetRoot();
        TreeNode * rootd = grp2.d->GetRoot();
        if (rootc != rootd) {
            res = Case_2_1(F,T1,F2,k,grp2.c->GetId(),grp2.d->GetId());
            return res;
        }
        TreeNode * tmp;
        vector<int> pedant_one;
        if ((tmp=IsOneDistance(grp2.a, grp2.b))!= NULL){ 
            pedant_one.push_back(tmp->GetId());
        }   // `\color{greencomments}若a,b之间只有1个悬挂节点`
        if ((tmp=IsOneDistance(grp2.c, grp2.d))!= NULL){ 
            pedant_one.push_back(tmp->GetId()); 
        }   // `\color{greencomments}若c,d之间只有1个悬挂节点`
        if (pedant_one.size() != 0) {
            res = Case_2_2(F,T1,F2,k,pedant_one);
            return res;
        }
        if (roota != rootc){
            TreeNode * Rab = F2->GetNodeById(lca.ask(grp2.a->GetId(),
                                                         grp2.b->GetId()));
            res = Case_2_3(F,T1,F2,k,grp2,Rab);
            return res;
        }
        res = Case_2_4(F,T1,F2,k,grp2);
    }
    return res;
}
\end{lstlisting}
\end{CJK}






































