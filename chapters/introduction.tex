% !Mode:: "TeX:UTF-8"
% !Mode:: "TeX:UTF-8"

\chapter{引言}
\section{研究背景}
分子系统学在生物学中是一个非常重要的研究领域,其中进化树,又称系统发生树,是用来研究一群物种进化历史的标准模型,是分子系统学中不可缺少的重要工具。由于生物进化中可能产生的杂交、基因水平转移、基因重组等网状事件,导致一个物种的基因可能来源于多个祖先。多年来对生物进化历史的研究发现,杂交事件分别在植物、鸟类、鱼类的种群中均有发生。自然的杂交事件在哺乳动物,甚至是灵长类动物中都被发现过。研究显示,25\%的植物和10\%的动物,尤其是年轻的物种,都与杂交事件相关。已有许多方法能够从一组物种中构建起他们对应的进化树,但由于进化过程中网络事件的存在,一组相同物种基于不同基因的分析可能产生不同的进化树,反之,通过对比这些进化树的相似性是帮助人们发现这些网络事件的重要手段。许多计算生物学家都对此问题非常感兴趣,他们常用的衡量标准有子树剪切再接距离(subtree prune and regraft distance, rSPR)和杂交数(hybridization number, HN)两种。Baroni 等人证明了rSPR距离为进化过程中网状事件的数目给出了一个下限。

\section{研究意义}

\section{近似算法}

\section{参数算法}
